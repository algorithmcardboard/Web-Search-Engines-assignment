\documentclass{article}
\usepackage{lmodern}
\usepackage{amsmath}
\usepackage{systeme}
\usepackage{amssymb}
\usepackage{listings}
\usepackage[T1]{fontenc}
\usepackage{fancyhdr}
\pagestyle{fancy}
\lhead{Himaja Rachakonda, Anirudhan J. Rajagopalan}

\begin{document}

\title{Web Search Engines --- Project Proposal}
\date{March 21, 2016}
\author{Himaja Rachakonda, Anirudhan J. Rajagopalan\\ N14633788, N18824115\\ hr970, ajr619}
\maketitle
\newpage

\section{Title}
The project is to be titled ``Review-Rehashed'' will summarize the reviews of a product with respect to a particular feature.

\section{Team Members}
\begin{enumerate}
  \item Himaja Rachakonda, N14633788, hr970@nyu.edu
  \item Anirudhan J. Rajagopalan, N18824115, ajr619@nyu.edu
\end{enumerate}

\section{Objective}
The aim of this project is to extract sentences from online reviews which discuss about the features of a product and present it to the user.  
For example for a search term, say, ``LG G3, Battery Life'' will list the excerpts from all reviews that discuss about the battery life of LG G3 phone. 
Depending on the computational complexity and performance of the search engines, we will try to expand the search feature to multiple features and combination of features.

\section{Sketch of Architecture}

\section{List of web resources}
We are planning to scrape data from 
\begin{enumerate}
  \item Amazon.com
  \item Bestbuy.com
\end{enumerate}
in very narrow categories of products (say electronics, or books).

\section{Technologies used}
We are planning to use the following resources for building the search engine.
\begin{description}
  \item[Programming Language]: Java, J2EE
  \item[Web server]: Apache Tomcat
  \item[Index Store]: Apache Lucene
  \item[NLP library]: Apache Mahout
\end{description}

\end{document}
